\JGWitem{Nov 2000, \#4}
 Amber, age 25, has a future lifetime random variable survival funtion given by $S_{25}(t)=1-t/75.$    % \S2 Single Life Survival
If she takes up skydiving for the next year, her assumed mortality will be such that for the coming year only, she will have a constant 
force of mortality of $0.1$.

Calculate the decrease in the 11-year temporary complete life expectancy for 
Amber if she takes up skydiving.
\showMC{\begin{description}
\item[(A)] 0.10
\item[(B)] 0.35
\item[(C)] 0.60
\item[(D)] 0.80
\item[(E)] 1.00
\end{description}}

\showsol{\bsoln
The n-year temporary complete life expectancy is 
\[
   \ecirc_{x:\halfbox{n}} = \int_{t=0}^{n} \prepostsub{t}{p}{x}\,dt.
\]
In this problem $x=25$ and $n=11$, and ``de Moivre'' means $S(x) = (\omega-x)/\omega$ which means $\prepostsub{t}{p}{x} = 1-t/(\omega-x)$, so that
\[
   \ecirc_{25:\halfbox{11}} = \int_{t=0}^{11} 1-t/75\,dt = 10.1933.
\]
Under the modification, for $t>1$ we have $\px[t]{[x]}=\px{[x]}\cdot(\px[(t-1)]{[x]+1})=\px{[x]}\cdot(\px[(t-1)]{x+1}),$ and so we have 
\bears
   \ecirc_{x:\halfbox{n}} = \int_{t=0}^{n} \prepostsub{t}{p}{[x]}\,dt 
     &=& \int_{t=0}^{1} \prepostsub{t}{p}{[x]}\,dt +\int_{t=1}^{n} \prepostsub{t}{p}{[x]}\,dt \\
     &=&    \int_{t=0}^{1} \prepostsub{t}{p}{[x]}\,dt  + \int_{t=1}^{n} \px{[x]}\cdot(\px[(t-1)]{x+1})\,dt \\
     &=&    \int_{t=0}^{1} \prepostsub{t}{p}{x}\,dt  + \px{[x]}\int_{t=0}^{n-1} \prepostsub{t}{p}{x+1}\,dt  \\
     &=&    \int_{t=0}^{1} e^{-0.1t}  \,dt  + e^{-0.1}\int_{t=0}^{n-1} 1-t/74\,dt  \\
     &=& 9.3886.
\eears
Difference is 0.8047

\esoln}
