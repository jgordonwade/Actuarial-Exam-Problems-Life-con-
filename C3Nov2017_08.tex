\JGWitem{ C3, Nov 2017 \#08}

\includegraphics[scale=0.8]{c:/Users/gwade/BGSUmath/Actuarial/MyCourse3stuff/SingleProblems/images/C3Nov2017_08.png}

\showsol{\bsoln
\bears
    0 = \Expect{L_0^g} &=& \overbrace{100,000\cdot \ax**[10|]{70}}^{\text{EPV of annuity pmts}}
      + \overbrace{0.7G + 0.05G\cdot \ax**{70:\angl{10}}}^{\text{EPV of expenses}} \; - \overbrace{G\cdot \ax**{70:\angl{10}}}^{\text{EPV of premiums}} \\
      0 &=& 100,000\cdot \ax**[10|]{70} - (0.95 \ax**{70:\angl{10}} - 0.7)G 
 \eears
The following values are from the ILT:
\[
   \ax**{70} = 8.5693, \qquad
   \ax**{80} = 5.9050, \qquad
   \Ex[10]{70} = 0.33037.
\]
From these we can compute
\bears
   \ax**[10|]{70} &=& \Ex[10]{70}\cdot \ax**{80}  =  0.33037 \cdot  5.9050 = 1.95083 \qquad\text{and} \\
   \ax**{70:\angl{10}} &=& \ax{70} - \ax**[10|]{70} =  8.5693 - 1.95083 = 6.61847
 \eears
 so that
 \[ 0 = 100,000\cdot 1.95083 - (0.95\cdot 6.61847 - 0.7)G, \]
so \fbox{G=34914}.

\esoln}

