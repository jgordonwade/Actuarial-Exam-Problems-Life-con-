\JGWitem{Nov 2012 \#05}
A modified annual two-year endowment insurance on (x) for 2,000 pays an {\em additional} death benefit equal to the
net premium policy value at the end of the first year.

For year 2, calculate the net premium policy value just {\em before} the maturity benefit is paid.
%Calculate the net premium policy value at a duration of two years, just before the maturity benefit is paid.

You are given:
\begin{enumerate}
\item $i = 0.10$
\item $q_x=0.150$ and $q_{x+1}=0.165$.
\end{enumerate}
Calculate the level annual benefit premium.

\showsol{\bsoln
This is a situation where comment \#5  after Def'n 2 is applicable. It says,
\begin{quote}
``If we are calculating a policy value at an integer duration, end of a year,
there may be premiums and/or expenses and/or benefits
payable at precisely that time and we need to be careful about which cash
flows are included in our future loss random variable. It is the usual practice
to regard a premium and any premium-related expenses due at that time as
future payments and any insurance benefits (i.e. death or maturity claims)
and related expenses as past payments.'' 
\end{quote}
%The death benefit --- call it $B_k$, is $B_k=2000+\prepostsub{k+1}{V}{}$. 

\medskip
We have $\prepostsub{0}{V}{}=0$ and $\prepostsub{2}{V}{}=2000.$
The first of these is because we are using ``benefit'' premium 
which means Equivalence Principle. The second of these is 
because just before the maturity benefit is paid, we have to have that amount on hand in reserve. 

\medskip
Denote by $\pi$ the level annual premium. If (x) fails in the first year, then the payout is $2000+\prepostsub{1}{V}{}$, and since 
$\prepostsub{0}{V}{} = 0 $ we have
\bears
   \left(\prepostsub{0}{V}{} + \pi\right)(i+1) = 1.1 \pi &=& q_x(2000+\prepostsub{1}{V}{}) + p_x\prepostsub{1}{V}{} \\
       1.1 \pi &=& \prepostsub{1}{V}{} + 300 \\
        \prepostsub{1}{V}{} &=& 1.1\pi -  300
\eears
If (x) fails in the second year, then the payout is $2000+\prepostsub{2}{V}{}$. Also, 
$\prepostsub{2}{V}{}=2000$. So,
\bears
   \left(\prepostsub{1}{V}{} + \pi\right)(i+1) &=& q_{x+1}(2000+\prepostsub{2}{V}{}) + \px{x+1}\cdot\left(_2\!V\right) \\\
   \left(\prepostsub{1}{V}{} + \pi\right) &=& \left(0.165\times 4000 + 0.835\times 2000\right)/1.1  = 2118.18 \\
   \left(1.1\pi - 300 + \pi\right) &=& 2118.18 \quad\Longrightarrow \underline{\pi = 1151.52}
\eears

\esoln}
