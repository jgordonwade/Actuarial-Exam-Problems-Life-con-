\JGWitem{May 2000, \#40}  % Markov chain
Rain is modeled as a Markov process with two states:
\begin{enumerate}
\item If it rains today, the probability that it rains tomorrow is $0.50.$
\item If  it does not rain today, the probability that it rains tomorrow is $0.30.$
\end{enumerate}
Calculate the limiting probability that it rains on two consecutive days.

\showsol{
\bsoln
Here's the idea. First, set up the ``stochastic matrix'' (or ``probability transition matrix'') $P$ for this problem. 
The find the state-state or long term probabilities, by finding the eigenvector corresponding to $\lambda =1 $.
From this extract the probability that it will be raining on a given day in the distant future. Then you can answer the question very simply.

\bigskip
\noindent\underline{Details:} \\
%\( X(n) = \left\{\begin{array}{cl} 0 & \text{ no rain on day $n$} \\
%                                                        1 & \text{ rain on day $n$}\
%                 \end{array}\right. \)
Denote by $\pi_{i,n}$ the probability that the system is in state $i$ on day $n$.

The given info means:
\bears
   \pi_{0,n+1} &=& \Prob{\text{No rain tomorrow}} \\
                    &=& \Prob{\text{No rain tomorrow}\mid \text{No rain today}}\times \Prob{\text{No rain today}} \\
                     && \;+ \;\Prob{\text{No rain tomorrow}\mid \text{Rain today}}\times \Prob{\text{Rain today}} \\
   \pi_{0,n+1} &=& 0.7\pi_{0,n} + 0.5\pi_{1,n} \\\
  \text{and similarly \ } \pi_{1,n+1} &=& 0.3\pi_{0,n} + 0.5\pi_{1,n} 
\eears
\bears
    \boldsymbol{\pi}_{n+1} &=& \boldsymbol{\pi}_n\left[\begin{array}{cc} 0.7 & 0.3 \\ 0.5 & 0.5 \end{array}\right] \\
    \boldsymbol{\pi} &=& \boldsymbol{\pi}\left[\begin{array}{cc} 0.7 & 0.3 \\ 0.5 & 0.5 \end{array}\right] \\
    \boldsymbol{0} &=& \boldsymbol{\pi}\left[\begin{array}{cc} -0.3 & 0.3 \\ 0.5 & -0.5 \end{array}\right] \\
     \boldsymbol{\pi} &=& \left[\frac{5}{8},\frac{3}{8}\right]
\eears
so in the distant future the chance of a rain on a given day is $\frac{3}{8}$. Finally (here's the ``very simply'' part), the
probability that it will also rain in the next day, given that it's raining on the one day, $1/2$, so the
probability of two consecutive days of rain is $\dfrac{3}{8}\cdot \dfrac{1}{2} = \dfrac{3}{16} = \boxed{0.1875}$.

\esoln}
