\JGWitem{May 2000, \#2}
 Lucky Tom finds coins on his way to work at a Poisson rate of 0.5 
coins/minute. The denominations are randomly distributed:

\begin{enumerate}
\item 60\% of the coins are worth 1;
\item 20\% of the coins are worth 5; and
\item 20\% of the coins are worth 10.
\end{enumerate}

Calculate the conditional expected value of the coins Tom found during 
his one-hour walk today, given that among the coins he found, exactly ten 
were worth 5 each.

%\showMC{\begin{description}
%\item (A) 108
%\item (B) 115
%\item (C) 128
%\item (D) 165
%\item (E) 180
%\end{description}}

\showsol{\bsoln
 0.5 coins per minute is 30 coins per hour. 
\[
   \lambda_1=.6*30=18, \quad\quad
   \lambda_5=.2*30=6, \quad\quad
   \lambda_{10}=.2*30=6
\]
In ``thinning'' a Poisson process, the thinned processes are independent
of each other so the fact that he found 10 nickels does not affect the 
number of pennies and dimes he found. 
\[
  \Expect{X}=1\cdot\Expect{N_1} + 5\cdot 10 + 10\cdot\Expect{N_10}
   = 18+50+60=128.
\]
\underline{Ans: (C)}

 This is a thinning problem. The total number $N$ of coins he finds in the
hour has a Poisson(30) distribution. Let $N_1$, $N_5$, and $N_{10}$ be
the number of coins worth 1, 5, and 10, resp. Then each of these
has a Poisson distribution, and $\lambda_1=0.6\cdot 30 = 18$, 
$\lambda_5=0.2\cdot 30 = 6$,  and $\lambda_{10}=0.2\cdot 30 = 6$.

$N_1$, $N_5$ and $N_{10}$ are all independent. 
The expected number of pennies he finds is 18, we know he found ten nickels, 
and the expected number of dimes he finds is 6, 
so the expected total is 
$18\cdot 1 + 10\cdot 5 + 6\cdot 10 = 18+50 + 60= 128$. 

\esoln}

