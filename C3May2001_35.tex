\JGWitem{May 2001, \#35}
You have calculated the actuarial present value of a last-survivor whole life insurance of
1 on (x) and (y). You assumed:
\begin{enumerate}
\item The death benefit is payable at the moment of death.
\item The future lifetimes of (x) and (y) are independent, and each life has a constant
  force of mortality with $\mu = 0.06 .$
\item $d = 0.05$
\end{enumerate}
Your supervisor points out that these are not independent future lifetimes. Each
mortality assumption is correct, but each includes a common shock component with
constant force 0.02.

Calculate the increase in the actuarial present value over what you originally calculated.
\showMC{\begin{description}
\item[(A)] 0.020
\item[(B)] 0.039
\item[(C)] 0.093
\item[(D)] 0.109
\item[(E)] 0.163
\end{description}}

\showsol{\bsoln
  Using the formula pattern ``\dsy{\overline{A} = \dfrac{\mu}{\mu+\delta}}'' (which is valid whenever $\mu$ and $\delta$ are constant).

  Without the common shock information:
  \bears
      \overline{A}_x &=& \overline{A}_y = \dfrac{0.06}{0.6+0.05} = \dfrac{6}{11} \\
      \overline{A}_{xy} &=& \dfrac{0.06+0.06}{0.6+0.06+0.05} = \dfrac{12}{17} \\
       \overline{A}_{\overline{xy}} &=& \overline{A}_x +\overline{A}_y - \overline{A}_{xy} = \dfrac{72}{187} 
   \eears

  With the common shock information, since $\mu_x$ includes the common shock, then the force of mortality $\mu_x^*$ to which (x) is
  subject but (y) is not satisfies $\mu_x=\mu_x^*+\lambda$ or $0.06=\mu_x^*+0.02$ or $\mu_x^* = 0.04$. Similarly $\mu_y^* = 0.04$. So,
  \[  \mu_{xy} = \mu_x^* + \mu_x^* + \lambda = 0.10. \]
  Now repeat the calculations above:  
   \bears
      \overline{A}_x &=& \overline{A}_y = \dfrac{0.06}{0.6+0.05} = \dfrac{6}{11}  \\
      \overline{A}_{xy} &=& \dfrac{0.1}{0.1+0.05} = \dfrac{2}{3} \\
       \overline{A}_{\overline{xy}} &=& \overline{A}_x +\overline{A}_y - \overline{A}_{xy} = \dfrac{14}{33}
   \eears
   Difference is \dsy{\dfrac{14}{33}-\dfrac{72}{187}  \approx \boxed{0.039216}}

\esoln}
