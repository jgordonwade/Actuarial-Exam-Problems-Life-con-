\JGWitem{May 2005 \#2}
For a double-decrement model:
\begin{enumerate}
\item $\prepostsubsuprime{t}{p}{x}{1}=1-t/60$ for $0\leq t \leq 60$, 
\item $\prepostsubsuprime{t}{p}{x}{2}=1-t/40$ for $0\leq t \leq 40$.
\end{enumerate}
Calculate $\mu_{40}^{(\tau)}(20)$.
\showMC{\begin{description}
\item[(A)] 0.025
\item[(B)] 0.038
\item[(C)] 0.050
\item[(D)] 0.063
\item[(E)] 0.075
\end{description}}
\showsol{\bsoln
One approach:
\bears
   \prepostsubsuprime{t}{p}{x}{\tau} &=& \left(\prepostsubsuprime{t}{p}{x}{1}\right)\left(\prepostsubsuprime{t}{p}{x}{2}\right) 
          = \frac{t^2 - 100t + 2400}{2400} \\
   \frac{-d}{dt}\prepostsubsuprime{t}{p}{x}{\tau} &=& \frac{t-50}{1200} \\
   \mu_{40}^{(\tau)}(t) &=& \frac{-d/dt\: \prepostsubsuprime{t}{p}{x}{\tau} }{\prepostsubsuprime{t}{p}{x}{\tau} }
          = \frac{2(t-50)}{(60-t)(40-t)} \\
   \mu_{40}^{(\tau)}(20) &=& \frac{3}{40} = 0.075
\eears
Another approach: For $j=1$ and $j=2$ we have 
   $\mu_{40}^{(j)}(t) = \left(-d/dt\: \prepostsubsuprime{t}{p}{x}{\tau} \right)/\left(\prepostsubsuprime{t}{p}{x}{\tau} \right)$
so 
\bears
   \mu_{40}^{(1)}(t) &=& \frac{1}{60-t} \quad\mbox{ so }\quad  \mu_{40}^{(1)}(20) = \frac{1}{40}  \\
   \mu_{40}^{(2)}(t) &=& \frac{1}{40-t} \quad\mbox{ so }\quad  \mu_{40}^{(2)}(20) = \frac{1}{20}  \\
   \mu_{40}^{(\tau)}(t) &=& \mu_{40}^{(1)}(t) + \mu_{40}^{(2)}(t) \quad\mbox{ so }\quad  \mu_{40}^{(\tau)}(20) = \frac{3}{40} = 0.075.
\eears
\esoln}



