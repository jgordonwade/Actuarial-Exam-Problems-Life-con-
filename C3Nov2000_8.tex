\JGWitem{Nov 2000, \#8}
 The number of claims, N, made on an insurance portfolio follows the following
distribution:

\begin{center}\begin{tabular}{|c|c|}\hline
$n$ & $\Prob{N=n}$ \\ \hline
0   & 0.7 \\
2   & 0.2 \\
3   & 0.1 \\ \hline
\end{tabular}\end{center}


If a claim occurs, the benefit is 0 or 10 with probability 0.8 and 0.2, respectively.
The number of claims and the benefit for each claim are independent.

\smallskip
Calculate the probability that aggregate benefits will exceed expected benefits by more
than 2 standard deviations.
\showMC{\begin{quote}\begin{description}
\item[(A)] 0.02
\item[(B)] 0.05
\item[(C)] 0.07
\item[(D)] 0.09
\item[(E)] 0.12
\end{description}\end{quote}}

\showsol{\bsoln
\bears
   \Expect {N} &=& 0.7 \\
   \Var{N} &=& 2^2\cdot 0.2 + 3^2\cdot 0.1 -(0.7)^2 = 1.21 \\
   \Expect{X} &=& 2 \\
   \Var{X} &=& 10^2\cdot 0.2 - (2)^2 = 16 \\
   \Var{S} &=& \Expect{N} \Var{X} + \Expect{X}^2 \Var{N} = 0.7 \cdot 16 + 4 \cdot 1.21= 16.04 \\
   \Std{S} &=& 4.005 \\
   \Expect{S} + 2\Std{S} &=& 1.4 + 2\cdot 4.005 = 9.4 \\
   \Prob{S>9.04} &=& 1-\Prob{S=0}  = 1 - \Bigl( 0.7  + 0.2\cdot 8^2 - 0.1\cdot 8^3\Bigr)  = 0.12.
\eears
\esoln}
