\JGWitem{May 2007 \#21}
You are given the following information about a new model for buildings with limiting age $\omega$.
\begin{enumerate}
\item The expected number of buildings surviving at age $x$ will be $l_x=(\omega-x)^a$ for $x<\omega$. 
\item The new model predicts a $1/3$ higher complete life expectency (over the previous deMoivre model with the same $\omega$) for buildings aged 30.
\item The complete life expectency for buildings aged 60 under the new model is 20 years. 
\end{enumerate}
Calculate the complete life expectency under the previous deMoivre model for buildings aged 70. 

\showsol{\bsoln
If $l_x=(\omega-x)^a$ for $x<\omega$ then
\bears
   \ecirc_x &=& 
\int_{t=0}^{\omega-x} \prepostsub{t}{p}{x}\,dt = \int_{t=0}^{\omega-x} \frac{l_{x+t}}{l_x}\,dt = \frac{1}{(\omega-x)^a} \int_{t=0}^{\omega-x} (\omega-x-t)^a\,dt \\
            &=& \frac{\omega-x}{a+1}
\eears
From the problem statement we have 
\bears
    \frac{4}{3} &=& \frac{\left[\frac{\omega-30}{a+1}\right]}{\left[\frac{\omega-30}{2}\right]} \quad\Longrightarrow\quad \underline{a+1 = 3/2} \\
    \ecirc_{60} &=& 20 = \frac{\omega-60}{a+1} = \frac{2}{3}\cdot(\omega-60) \quad\Longrightarrow\quad \underline{\omega = 90} \\
    \ecirc_{70} &=& (90-70)/2 = 10.
\eears

\esoln}
