\JGWitem{Nov 2001, \#12}
A fund is established by collecting an amount $P$ from each of 100 
independent lives age 70. The fund will pay the following benefits:
\begin{description}
\bull 10, payable at the end of the year of death, for those who die before 
  age 72, or
\bull $P$, payable at age 72, to those who survive.
\end{description}
You are given:
\begin{enumerate}
 \item Mortality follows the Illustrative Life Table.
 \item $i = 0.08$
\end{enumerate}
 Calculate $P$, using the equivalence principle.
\showMC{\begin{description}
\item[(A)] 2.33
\item[(B)] 2.38
\item[(C)] 3.02
\item[(D)] 3.07
\item[(E)] 3.55
\end{description}}
\showsol{\bsoln
The EPV of the benefit for an individual in this group is
\[
    10\Ax{\term{70}{2}}+P\Ex[2]{70} 
\]
and since each of the 100 lives-age-70 is independent, the EPV for the whole group is just 100 time this amount. 
So the equivalence principle gives
\bears
   0 &=& 1000 \Ax{\term{70}{2}}+100P\Ex[2]{70}  - 100P \\
   0 &=& 10 \Ax{\term{70}{2}}-P\left(1-\Ex[2]{70}\right) \\
   P &=& \dfrac{10 \Ax{\term{70}{2}}}{1-\Ex[2]{70}}
\eears
\esoln}
