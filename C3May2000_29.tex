\JGWitem{May 2000, \#29}
 For a whole life annuity-due of 1 on (x), payable annually:
\begin{enumerate}
\item $q_x = 0.01$
\item $q_{x+1} = 0.05$
\item $i = 0.05$
\item $\ddot{a}_{x+1}=6.951$
\end{enumerate}
Calculate the change in the actuarial present value of this annuity-due if 
$p_{x+1}$ is increased by $0.03$.
\showMC{\begin{description}
\item[(A)] 0.16
\item[(B)] 0.17
\item[(C)] 0.18
\item[(D)] 0.19
\item[(E)] 0.20
\end{description}}
\leaveit{\bsoln
Note that
\bears
   \ddot{a}_x &=& 1 + vp_x + \prepostsub{2}{E}{x}\ddot{a}_{x+2} 
   = 1 + vp_x + v^2p_x p_{x+1}\cdot\ddot{a}_{x+2}.
\eears
If $p_{x+1}$ is replaced by $p_{x+1}+0.03$ then the new APV, call it $\ddot{a}'_{x}$, is
\bears
   \ddot{a}'_x &=& 1 + vp_x + v^2p_x (p_{x+1}+0.03)\cdot\ddot{a}_{x+2} \\
    &=& \ddot{a}_x + 0.03\times v^2p_x\cdot\ddot{a}_{x+2},
\eears
so the change in the APV, call it $\Delta$-APV, is 
\bears
  \Delta\rm{-APV} &=& 0.03\times v^2p_x\cdot\ddot{a}_{x+2}.
\eears
Now, $\ddot{a}_{x+1} = 1 + vp_{x+1}\ddot{a}_{x+2}$. Solving this for $\ddot{a}_{x+2}$ and using it (and the other given data) in $\Delta$-APV, we have:
\bears
   \Delta\rm{-APV} &=& 0.03\times v^2p_x\cdot\ddot{a}_{x+2} \\
    &=& 0.03 \times vp_x(\ddot{a}_{x+1}-1)/p_{x+1} = \underline{0.177188}
\eears


 

\showMC{\bf Ans: (C)}

\esoln}
