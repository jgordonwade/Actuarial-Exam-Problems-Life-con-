\JGWitem{C3 May 2013 \#13}  % LTAM Sample Q's (Nov 2019) \#8.7
An automobile insurance company classifies its insured drivers into three risk categories. 
The risk categories and expected annual claim costs are as follows:
\begin{center}\begin{tabular}{cc} Risk Category & Expected Annual Claim Cost \\ \hline  Low & 100 \\ Medium & 300 \\ High & 600 \end{tabular}\end{center}
The pricing model assumes:
\begin{itemize}
\item At the end of each year, 75\% of insured drivers in each risk category will renew their insurance.
\item $i=0.06.$
\item All claim costs are incurred mid-year.
\end{itemize}
For those renewing, 70\% of Low Risk drivers remain Low Risk, and 30\% become Medium Risk,
40\% of Medium Risk drivers remain Medium Risk, 20\% become Low Risk, and 40\% become High Risk. 
All High Risk drivers remain High Risk.

\medskip

Today the Company requires that all new insured drivers be Low Risk. The present value of expected claim costs for the 
first three years for a Low Risk driver is 317. Next year the company will allow 10\% of new 
insured drivers to be Medium Risk.

\medskip
Calculate the percentage increase in the present value of expected claim costs for the first three years per new insured driver due to the change.
\showsol{\bsoln
\[ \begin{array}{c} \text{L} \\  \text{M} \\  \text{H} \end{array}
   \qquad Q = \left[\begin{array}{ccc} 0.7 & 0.3 & 0 \\ 0.2 & 0.4 & 0.4 \\ 0 & 0 & 1 \end{array}\right] 
   \qquad Q^2 = \left[\begin{array}{ccc} 0.55 & 0.33 & 0.12 \\ 0.22 & 0.22 & 0.56 \\ 0 & 0 & 1 \end{array}\right]   
\]
%  Here are the calculations for what we'll call $C_{L},$ the EPV of claims costs for someone who starts out in the Low Risk class.
%\begin{center}\begin{tabular}{ccl|c} Path & ${\rm Prob}\{\text{Path}\}$ & \rule{10pt}{0pt} EPV$\{$claims$|$path$\}$ & product \\ \hline
%  \rule{0pt}{22pt}$\text{L}\rightarrow\text{L}\rightarrow{L}$  & $0.7\cdot 0.7$ & $\dfrac{100}{1.06^{1/2}}$+$\dfrac{100\cdot 0.75}{1.06^{3/2}}$+$\dfrac{100\cdot 0.75^2}{1.06^{5/2}}$ & 105.093 \\
%  \rule{0pt}{22pt}$\text{L}\rightarrow\text{L}\rightarrow{M}$  & $0.7\cdot 0.3$ & $\dfrac{100}{1.06^{1/2}}$+$\dfrac{100\cdot 0.75}{1.06^{3/2}}$+$\dfrac{300\cdot 0.75^2}{1.06^{5/2}}$ & 65.4625 \\
%  \rule{0pt}{22pt}$\text{L}\rightarrow\text{M}\rightarrow{L}$ & $0.3\cdot 0.2$ & $\dfrac{100}{1.06^{1/2}}$+$\dfrac{300\cdot 0.75}{1.06^{3/2}}$+$\dfrac{100\cdot 0.75^2}{1.06^{5/2}}$  & 21.1154 \\
%  \rule{0pt}{22pt}$\text{L}\rightarrow\text{M}\rightarrow{M}$ & $0.3\cdot 0.4$ & $\dfrac{100}{1.06^{1/2}}$+$\dfrac{300\cdot 0.75}{1.06^{3/2}}$+$\dfrac{300\cdot 0.75^2}{1.06^{5/2}}$  & 53.9007 \\
%  \rule{0pt}{22pt}$\text{L}\rightarrow\text{M}\rightarrow{H}$ & $0.3\cdot 0.4$  & $\dfrac{100}{1.06^{1/2}}$+$\dfrac{300\cdot 0.75}{1.06^{3/2}}$+$\dfrac{600\cdot 0.75^2}{1.06^{5/2}}$  & 71.4056 \\ 
%   \hline && \rule{0pt}{14pt}\rule{60pt}{0pt} Sum: & 316.97
%\end{tabular}\end{center}
%This is what the problem statement gave us, so we actually didn't need to compute it. 
%
%\medskip
%Similarly, 
Here are the calculations for what we'll call $C_{M},$ the EPV of claims costs for someone who starts out in the Medium Risk class.
\begin{center}\begin{tabular}{ccl|c} Path & ${\rm Prob}\{\text{Path}\}$ & \rule{10pt}{0pt} EPV$\{$claims$|$path$\}$ & product \\ \hline
  \rule{0pt}{22pt}$\text{M}\rightarrow\text{L}\rightarrow{L}$  & $0.2\cdot 0.7$ & $\dfrac{300}{1.06^{1/2}}$+$\dfrac{100\cdot 0.75}{1.06^{3/2}}$+$\dfrac{100\cdot 0.75^2}{1.06^{5/2}}$ & 57.2227 \\
  \rule{0pt}{22pt}$\text{M}\rightarrow\text{L}\rightarrow{M}$  & $0.2\cdot 0.3$ & $\dfrac{300}{1.06^{1/2}}$+$\dfrac{100\cdot 0.75}{1.06^{3/2}}$+$\dfrac{300\cdot 0.75^2}{1.06^{5/2}}$ & 30.3590\\ 
  \rule{0pt}{22pt}$\text{M}\rightarrow\text{M}\rightarrow{L}$ & $0.4\cdot 0.2$ & $\dfrac{300}{1.06^{1/2}}$+$\dfrac{300\cdot 0.75}{1.06^{3/2}}$+$\dfrac{100\cdot 0.75^2}{1.06^{5/2}}$  & 43.6944 \\
  \rule{0pt}{22pt}$\text{M}\rightarrow\text{M}\rightarrow{M}$ & $0.4\cdot 0.4$ & $\dfrac{300}{1.06^{1/2}}$+$\dfrac{300\cdot 0.75}{1.06^{3/2}}$+$\dfrac{300\cdot 0.75^2}{1.06^{5/2}}$  & 102.949 \\
  \rule{0pt}{22pt}$\text{M}\rightarrow\text{M}\rightarrow{H}$ & $0.4\cdot 0.4$  & $\dfrac{300}{1.06^{1/2}}$+$\dfrac{300\cdot 0.75}{1.06^{3/2}}$+$\dfrac{600\cdot 0.75^2}{1.06^{5/2}}$  & 126.289 \\
  \rule{0pt}{22pt}$\text{M}\rightarrow\text{H}\rightarrow{H}$ & $0.4\cdot 1.0$  & $\dfrac{300}{1.06^{1/2}}$+$\dfrac{600\cdot 0.75}{1.06^{3/2}}$+$\dfrac{600\cdot 0.75^2}{1.06^{5/2}}$  &  398.189 \\
   \hline && \rule{0pt}{14pt}\rule{60pt}{0pt} Sum: & 758.703
\end{tabular}\end{center}

So if 90\% of the portfolio starts out at Low risk and 10\% starts out at Medum risk, then the expected claims will be $0.9\times 316.97 + 0.1 \times  758.703 = 361.1433.$

And since $361.1433/316.97 = 1.139$, the increase is \fbox{13.9\%}

% 0.2\cdot 0.7\cdot (\dfrac{300}{1.06^{1/2}}+\dfrac{100\cdot 0.75}{1.06^{3/2}}+\dfrac{100\cdot 0.75^2}{1.06^{5/2}})
% 0.2\cdot 0.3\cdot (\dfrac{300}{1.06^{1/2}}+\dfrac{100\cdot 0.75}{1.06^{3/2}}+\dfrac{300\cdot 0.75^2}{1.06^{5/2}})
% 0.4\cdot 0.2\cdot (\dfrac{300}{1.06^{1/2}}+\dfrac{300\cdot 0.75}{1.06^{3/2}}+\dfrac{100\cdot 0.75^2}{1.06^{5/2}})
% 0.4\cdot 0.4\cdot (\dfrac{300}{1.06^{1/2}}+\dfrac{300\cdot 0.75}{1.06^{3/2}}+\dfrac{300\cdot 0.75^2}{1.06^{5/2}})
% 0.4\cdot 0.4\cdot (\dfrac{300}{1.06^{1/2}}+\dfrac{300\cdot 0.75}{1.06^{3/2}}+\dfrac{600\cdot 0.75^2}{1.06^{5/2}})
% 0.4\cdot 1.0\cdot (\dfrac{300}{1.06^{1/2}}+\dfrac{600\cdot 0.75}{1.06^{3/2}}+\dfrac{600\cdot 0.75^2}{1.06^{5/2}})

\esoln} 
