\JGWitem{May 2005 \#38}
 A group of 1000 lives each age 30 sets up a fund to pay 1000 at the end of the first year for
each member who dies in the first year, and 500 at the end of the second year for each
member who dies in the second year. Each member pays into the fund an amount equal to
the single benefit premium for a special 2-year term insurance, with:
\begin{enumerate}
\item Benefits:
  \begin{center}\begin{tabular}{c|c}
    $k$ & $b_{k+1}$ \\ \hline
    0 & 1000 \\
    1 & 500 
  \end{tabular}\end{center}
\item Mortality follows the ILT
\item $i = 0.06$
\end{enumerate}
The actual experience of the fund is as follows:
  \begin{center}\begin{tabular}{ccc}
   $k$ & Interest Rate Earned & Number of Deaths \\ \hline
   0 & 0.070 & 1 \\
   1 & 0.069 & 1 
  \end{tabular}\end{center}
Calculate the difference, at the end of the second year, between the expected size of the fund
as projected at time 0 and the actual fund.
\showsol{\bsoln
From the tables we have $\qx{30}=0.00152894$ and $\qx{31}=0.00160886$,
so that
$\qx[0|]{30}=0.00152894$, and $\qx[1|]{30} = (1-0.00152894)\cdot 0.00160886 = 0.00160640.$
So the EPV of th insurance is
\[   \dfrac{1000}{1.06}\cdot  0.00152894 + \dfrac{500}{1.06^2} \cdot 0.00160640 = 2.15724,\]
so the total amount collected is $\$2157.24$. The expected amount of this remaining after 2 years is zero.


In the first year this amount grows to $\$2157.24\cdot 1.07=2308.25$ and then \$1,000 is removed leaving \$1308.25.

In the second year this amount grows to $\$1308.25\cdot 1.069 = 1398.52$ and then \$500 is removed leaving \$898.52.

So the difference, at the end of the second year, between the expected size of the fund as projected at time 0 and the actual fund is \$898.52.

\esoln}


 




 
