\JGWitem{Nov 2012 \#18}
For a special whole life insurance on (40), you are given:
\begin{enumerate}
\item The death benefit is payable at the end of the year of death, and is 1000 during the first 11 years and 5000 thereafter.
\item Expenses, payable at the beginning of the year, are 100 in year 1 and 10 in years 2 and later.
\item $\pi$ is the level annual premium, determined using the equivalence principle.
\item $G = 1.02 \pi$ is the level annual gross premium.
\item Mortality follows the Illustrative Life Table.
\item $i = 0.06$
\end{enumerate}
Calculate the gross premium policy value at the end of year 1 for this insurance.

\showsol{\bsoln
% APV of benefits at issue: $1000 A_{40}  + 4000\,\prepostsub{11|}{A}_{40} =  683.965$
\\
The Equivalence Principle implies
\bears
   0 &=& 1000\cdot \Ax{40} + 4000\Ex[11]{40}\Ax{51} + 90 + 10\ax**{40} - \pi \ax**{40} \\
  &=&  683.9648 + 90  + 10\cdot 14.816605 - 14.816605\cdot \pi 
    \quad\Longrightarrow \quad \underline{\pi = 62.2363.}
\eears

Now the required policy value is
\bears
   _1\!V &=& 1000\cdot \Ax{41} + 4000\Ex[10]{41}\Ax{51} + 10\ax**{41} - 1.02\cdot 62.2363 \ax**{41}  \\
    &=& 168.6916 + 555.5441 + 146.8645 - 932.3108 =  \boxed{-61.21}
\eears

\esoln}
