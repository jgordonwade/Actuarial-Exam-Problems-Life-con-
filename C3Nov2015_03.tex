\JGWitem{C3 Nov 2015 \#03}  %LTAM Sample(Nov 2019) #8.18
Johnny Vegas performs motorcycle jumps throughout the year and has injuries in the course of his shows according to the following three-state Markov model:
\begin{description}
\item State 0: No injuries
\item State 1: Exactly one injury
\item State 2: At least two injuries
\end{description}
You are given:
\begin{itemize}
\item[(i)] Transition intensities between States are per year.
\item[(ii)] $\mu_t^{01} = 0.03 +0.06\cdot 2^t$
\item[(iii)] $\mu_t^{02} = 2.718\mu_t^{01}$
\item[(iv)] $\mu_t^{12} = 0.025$
\end{itemize}
Calculate the probability that Johnny, who currently has no injuries, will sustain at least one injury in the next year.
\showsol{\bsoln
\[   \text{ANS} = 1-\px{}[\overline{00}] = 1-\exp\left(-\int_0^1 (1+2.718)\cdot\left(0.03 +0.06\cdot 2^t\right) \,dt \right) = 0.351684 \]
Note: The sum of those hazard rates 
 is a form of Makeham's law and so the result could be calculated using the tables provided on the Exam. 
\esoln}  

