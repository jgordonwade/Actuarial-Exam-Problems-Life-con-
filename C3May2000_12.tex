\JGWitem{May 2000, \#12}
 For a certain mortality table, you are given:
\begin{description}
\item[(i)] $\mu(80.5) = 0.0202$
\item[(ii)] $\mu(81.5) = 0.0408$
\item[(iii)] $\mu(82.5) = 0.0619$
\item[(iv)] Deaths are uniformly distributed between integral ages.
\end{description}

Calculate the probability that a person age 80.5 will die within two years.
\showMC{\begin{description}
\item[(A)] 0.0782
\item[(B)] 0.0785
\item[(C)] 0.0790
\item[(D)] 0.0796
\item[(E)] 0.0800
\end{description}}

\showsol{\bsoln
 We use
  \begin{itemize}
     \item the fact that under UDD, for $x$ integers and $s\in [0,1]$ we have
         \be \label{C3May2000.12.c} \qx[s]{x} = s\cdot \qx{x} \ee and
         \be \label{C3May2000.12.b} \mu_{x+s}\dfrac{\qx{x}}{1-s\cdot \qx{x}}, \ee and
     \item  the principle \be \label{C3May2000.12.a} \qx[s+t]{x} = \qx[s]{x} + (\px[s]{x}(\qx[t]{x+s}) \ee
          which arises from the fact that $\{T_x\leq s+t\}$ is the union of two disjoint events, $\{T\leq s\}$ and $\{T_x\in(s,s+t]\}.$
   \end{itemize}

   From (\ref{C3May2000.12.b}) and the given problem data we obtain
   \[  \qx{80} = 0.02, \quad\quad \qx{81}=0.04, \quad\text{and} \quad \qx{82}=0.06. \]
   Then
   \bears
       \qx[2]{80.5} &=& \qx[0.5]{80.5} + (\px[0.5]{80.5})\cdot ( \qx[1.5]{81}) \quad\text{from (\ref{C3May2000.12.a})} \\
       \qx[2]{80.5} &=& \qx[0.5]{80.5} + (\px[0.5]{80.5})\cdot \bigl[ \qx{81} + (\px{81})(\qx[0.5]{82})\bigr]
              \quad\text{from (\ref{C3May2000.12.a}) again.} \\
   \eears
   Now $\px[0.5]{80.5} = (\px{80})/(\px[0.5]{80})= (\px{80})/(1-\qx[0.5]{80}) = (\px{80})/(1-0.5\cdot \qx{80})
   =0.98/0.99$, so 
   \bears
          \qx[2]{80.5} &=& \dfrac{0.01}{0.99} + \dfrac{0.98}{0.99}\cdot \left[ 0.04 + (0.96)(0.5\cdot 0.06)\right] \\
              &=& \boxed{0.07821}
   \eears
\esoln}
