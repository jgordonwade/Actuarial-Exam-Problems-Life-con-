\JGWitem{C3 May 2007, 10}
For whole life insurances of 1000 on (65) and (66):
\begin{enumerate}
 \item Death benefits are payable at the end of the year of death.
 \item The interest rate is 10\% for 2008 and 6\% for 2009 and thereafter.
 \item $q_{65} = 0.010$ and $q_{66} = 0.012$. 
 \item The actuarial present value  on December 31$^{st}$, 2007, of the insurance on (66) is 300.
\end{enumerate}
Caclulate the actuarial present value  on December 31$^{st}$, 2007, of the insurance on (65).
\showMC{\begin{description}
\item[(A)] 279 
\item[(B)] 284 
\item[(C)] 289 
\item[(D)] 293 
\item[(E)] 298 
\end{description}}


\showsol{\bsoln
Because of the differing interest rate, we have to go to first-principles. Denote by $1000A^*_{65}$ and $1000A^*_{66}$ the APV's of the two insurances at the end of 2007, and
let $A_x$ have it's usual meaning, computed at an interest rate of 6\%. We have:
\bears
  \frac{300}{1000} = A_{66}^* &=& vq_{66} + vp_{66}A_{67} \\
  \frac{3}{10} = &=& \frac{0.010}{1.1} + \frac{1-0.012}{1.1}A_{67}  \\
  \Longrightarrow\quad A_{67}&=& 0.32186 \\
  A_{66} &=& vq_{66} + p_{66}A_{67} \\
       &=& \frac{0.012}{1.06} + \frac{1-0.012}{1.06}\cdot 0.32186 = 0.3113 \\
  A^*_{65} &=& vq_{65} + p_{65}A_{66} \\
          &=& \frac{0.010}{1.1} + \frac{1-0.010}{1.06}\cdot 0.3113 = 0.289
\eears

\esoln}
