\JGWitem{C3 Nov 2005, \#32}   % \S2 Single Life Survival
For a group of lives aged 30, containing an equal number of smokers and non-smokers, you are given:
\begin{enumerate}
  \item For non-smokers, $\mu^{n}(x)=  0.08$ for $x\geq 30$, 
  \item For smokers, $\mu^{s}(x)=  0.16$ for $x\geq 30$. 
\end{enumerate}
Calculate $q_{80}$ for a life randomly selected from those surviving to age 80.

\medskip{\it Hint:} Initially the ratio of smokers/nonsmokers is  known (50/50). How does this ratio change over time?
  Initially (at age 30), there are equal numbers of non-smokers and smokers. What's the ratio by age 80?

\showsol{\bsoln
  \bears
     \mbox{chance of non-smoker reaching age 80 } = \prepostsub{50}{p^{n}}{30} &=& e^{-0.08\cdot 50}  \\
     \mbox{chance of smoker reaching age 80 } = \prepostsub{50}{p^{s}}{30} &=& e^{-0.16\cdot 50} \\   \\ 
    %
     \frac{\mbox{chance of non-smoker reaching age 80 }}{\mbox{chance of smoker reaching age 80 }} &=& e^{0.08\cdot 50} = e^4.
  \eears
  So the ratio of non-smokers to smokers, amoung those surviving to age 80, is $e^4:1$. So
  \bears
      \Prob{\mbox{Selecting a non-smoker}} = \frac{e^4}{e^4+1}, \\
      \Prob{\mbox{Selecting a smoker}} = \frac{1}{e^4+1}
   \eears
  \bigskip
  Now just take a weighted average:
  \bears
     \mbox{$q_{80}$ for a randomly selected (80)} &=& \frac{e^4\cdot (1-e^{-0.08}) + 1\cdot (1-e^{-0.16})}{e^4+1} = \boxed{0.07816}
   \eears
\esoln}


%\showsol{\bsoln
%   For a given, fixed value of the force of mortality $\mu$, the probability that a life-age-30 survives to age 80 but not to age 81 is
%\[   \qx[50|]{30} = \px[50]{30}\qx{80} = e^{-50\mu}(1-e^{-\mu}) = e^{-50\mu}-e^{-51\mu}) \]
%so 
%\bears
%   \qx[50|]{30}[n] &=& e^{-50\cdot 0.08} - e^{-51\cdot 0.08} = 0.00140817 \qquad\text{for non-smokers} \\
%   \qx[50|]{30}[s] &=& e^{-50\cdot 0.16} - e^{-51\cdot 0.16} = 0.0000496002  \qquad\text{for smokers}, \\
%\eears
%and so
%( 0.00140817 + 0.0000496002)/2 --- WRONG
%\esoln}
