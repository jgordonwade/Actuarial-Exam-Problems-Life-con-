\JGWitem{Nov 2001, \#4}
There are two different insurance whole life policies on (x) with a death benefit of 5000, policy (I) and policy (II).
Policy (I) pays pays at the end of the year of death, and policy (II) pays at at the end of the year of death
\underline{if} (x) survives at least one year.

For each policy, the contract premium is the annual net level premium.

Additionally you are given:
\begin{enumerate}
\item $q_x = 0.05$
\item $v = 0.90$
\item $\ddot{a}_x  = 5.00$
\item The net premium policy value $_{10}\!V^{(I)}$ for policy (I) is $0.20000.$ 
\end{enumerate}
Calculate the net premium policy value $_{10}\!V^{(II)}$ for policy (II).
\showMC{\begin{description}
\item[(A)] 795
\item[(B)] 1000
\item[(C)] 1090
\item[(D)] 1180
\item[(E)] 1225
\end{description}}
\showsol{\bsoln
Prospectively, for $t>1$, 
\[
    \prepostsub{10}{V} \  = 5000\cdot A_{x+10} - \pi\ddot{a}_{x+10}.
\]
The net annual premium for this policy, call it $\pi$, is
\[  \pi = \dfrac{5000\cdot v\cdot p_x\cdot A_{x+1}}{\ddot{a}_{x}}. \]
There is enough information given to allow us to compute the following:
\begin{description}
 \item $A_x$ 
 \item $\pi$
\end{description}

Also, for a fully discrete whole life policy on (x) with sum-insured of 1, the premium $P_x$ is satisfies
\(   0 = A_x + P_x\ddot{a}_x. \) There's enough information given to allow us to compute $P_x$.

\medskip

The benefit reserve for that policy after 10  years, $\prepostsub{10}{V}{x}$,
is  \[ 
\prepostsub{10}{V}{x} = A_{x+10} - P_x \ddot{a}_{x+10},
\] and so there enough information
to compute $A_{x+10}$ and $\ddot{a}_{x+10}$. 

\medskip
\[
    \prepostsub{10}{V} = 5000\cdot A_{x+10} - \pi\ddot{a}_{x+10} = 5000\cdot 0.6 - 455\cdot 4 = \boxed{1180}
\]

\esoln}
  
