\JGWitem{May 2005 \#37}
 Company ABC sets the contract premium for a continuous life annuity of 1 per year on (x)
equal to the single benefit premium calculated using:
\begin{enumerate}
\item $\delta = 0.03$
\item $\mu_x(t) = 0.02$ for $t \geq 0$
\end{enumerate}
However, a revised mortality assumption reflects future mortality improvement and is given by
\[
   \mu_x(t) = \left\{\begin{array}{cc} 
      0.02 & \mbox{ for }t\leq 10 \\
      0.01 & \mbox{ for }t>    10 \\
   \end{array}\right.
\]
Calculate the expected loss at issue for ABC (using the revised mortality assumption) as a
percentage of the contract premium.
\showMC{\begin{description}
\item[(A)] 2\%
\item[(B)] 8\%
\item[(C)] 15\%
\item[(D)] 20\%
\item[(E)] 23\%
\end{description}}

\showsol{\bsoln
The ``single benefit premium'' for a life annuity is the EPV of that annuity, which is $\ax*{x}:$
\[  \ax*{x} = \int_0^\infty e^{-\delta t}\px[t]{x}\,dt = \int_0^\infty e^{-\delta t}e^{-\mu t}\,dt
     = \dfrac{1}{\mu+\delta} = \dfrac{1}{0.05} = 20. \]

Under the revised mortality assumptions, the expected present value --- call it $a$ --- of the life annuity on (x) can be viewed
as the sum of a 10-year temporary life annuity on (x), \underline{plus} a deferred whole life annuity on (x+10). So
\[
    a = \ax*{\endow{x}{10}} + \Ex[10]{x}\ax*{x+10} = \int_0^{10} e^{-0.05 t}\,dt + e^{-0.05\cdot 10}\int_0^{\infty} e^{-0.04 t}\,dt  = 23.0327
\]
So we lose 3.0327 in this policy. As a percentage of the contract premium, this is $(100\cdot 3.0327/20)\%$ or \fbox{15.16\%}.

\esoln}
