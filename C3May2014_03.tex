\JGWitem{ C3 May 2014 \#03} %LTAM Sample Q's (Nov 2019) \#8.11
A continuous Markov process is modeled by the following multiple state diagram:
\[   \text{State 0}\qquad\text{State 1}\qquad\text{State 2} \]
You are given the following constant transition intensities:
\begin{itemize}
\item[(i)] $\mu^{01} = 0.08$
\item[(ii)]  $\mu^{02} = 0.04$
\item[(iii)]  $\mu^{10} = 0.10$
\item[(iv)]  $\mu^{12} = 0.05$
\end{itemize}
For a person in State 1, calculate the probability that the person will continuously remain in State 1 for the next 15 years.
\showsol{\bsoln
\bears
  \dfrac{d}{dt}\actsymb[t]{\mathbf{P}} \ &=& \actsymb[t]{\mathbf{P}} \ \mathbf{U} = 
  \left[\begin{array}{ccc} \px[t]{}[00] & \px[t]{}[01] & \px[t]{}[02]  \\  \px[t]{}[10] & \px[t]{}[11] & \px[t]{}[12]  \\  \px[t]{}[20] & \px[t]{}[21] & \px[t]{}[22] \end{array}\right]
                                     \left[\begin{array}{ccc} -0.12 & 0.08 & 0.04 \\ 0.10 & -0.15 & 0.05 \\ 0 & 0 & 0 \end{array}\right] \\
  \dfrac{d}{dt}  \px[t]{}[11] &=& -0.15\cdot \px[t]{}[11] , \qquad  \px[0]{}[11]=1,  \\
  \px[t]{}[\overline{11}] &=& e^{-0.15\cdot t} \\
  \px[15]{}[\overline{11}] &=& e^{-0.15\cdot 15} = \boxed{0.105399}
\eears
\esoln}
