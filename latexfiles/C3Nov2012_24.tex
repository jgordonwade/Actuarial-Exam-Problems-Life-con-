\JGWitem{C3 Nov 2012 \#24}  % LTAM Sample (Nov 2019) #8.
An insurance company is designing a special 2-year term insurance. Transitions are modeled as a four-state homogeneous
Markov model with states:
\begin{description}
\item (H) Healthy
\item (Z) infected with virus ``Zebra''
\item (L) infected with virus ``Lion''
\item (D) Death
\end{description}
The annual transition probability matrix is given by:
\[ \begin{array}{c} \text{H} \\ \text{Z} \\ \text{L} \\ \text{D} \end{array}
   \left[\begin{array}{cccc} 
      0.90 & 0.05 & 0.04 & 0.01 \\ 0.10 & 0.20 & 0 & 0.70 \\ 0.20 & 0 & 0.20 & 0.60 \\ 0 & 0 & 0 & 1 \end{array}\right] \]
  %Matlab A=[  0.90   0.05   0.04   0.01 ; 0.10   0.20   0   0.70 ; 0.20   0   0.20   0.60 ; 0   0   0   1 ]
You are given:
\begin{itemize}
\item[(i)] Only one transitions can occur during any given year.
\item[(ii)] 250 is payable at the end of the year in which you become infected with either virus.
\item[(iii)] For lives infected with either virus, 1000 is payable at the end of the year of death.
\item[(iv)] The policy is issued only on Healthy lives.
\item[(v)] $i=0.05$
\end{itemize}
Calculate the actuarial present value of the benefits at policy issue.

\showsol{\bsoln
\begin{center}\begin{tabular}{cccc}
Possible transitions & probability & discounted benefits & APV \\ \hline
H $\rightarrow$Z & 0.05 & 250$v$ &11.904762 \\ 
H $\rightarrow$L & 0.04 & 250$v$ &9.523810 \\
H $\rightarrow$Z $\rightarrow$ D & 0.05(0.7) = 0.035 & 1000$v^2$ & 31.746032 \\
H $\rightarrow$L$\rightarrow$ D & 0.04(0.6) = 0.024 & 1000$v^2$ & 21.768707 \\
H $\rightarrow$H $\rightarrow$ Z & 0.90(0.05) = 0.045 & 250$v^2$ & 10.204082 \\
H $\rightarrow$H $\rightarrow$ L & 0.90(0.04) = 0.036 & 250$v^2$ & 8.163265
\end{tabular}\end{center}
Sum of these gives total APV =93.31066
\esoln}


