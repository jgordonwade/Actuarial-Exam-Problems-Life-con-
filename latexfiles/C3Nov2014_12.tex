\JGWitem{C3 Nov 2014 \#12} %LTAM Sample Q's (Nov 2019) \#8.15
You are analyzing the sensitivity of some of the assumptions used in setting the premium rate for a sickness policy.
 You are basing your calculations on a multiple state model as diagrammed below:
\begin{center} Healthy { \ \ } Sick { \ \ } Dead \end{center}
You are given:
\begin{itemize}
\item[(i)] Level premiums are paid continuously by Healthy policyholders.
\item[(ii)] Level sickness benefits are paid continuously to Sick policyholders.
\item[(iii)] There is no death benefit.
\end{itemize}
Which one of the following changes to the assumptions will be certain to increase the premium rate?
\begin{itemize}
\item[(A)] A lower rate of interest and a higher recovery rate from the Sick state.
\item[(B)] A lower mortality rate for those in the Healthy state and a lower mortality rate for those in the Sick state.
\item[(C)] A higher mortality rate for those in the Healthy state and a higher mortality rate for those in the Sick state.
\item[(D)] A lower recovery rate from the Sick state and a lower mortality rate for those in the Sick state.
\item[(E)] A higher rate of interest and a lower mortality rate for those in the Healthy state.
\end{itemize}
\showsol{\bsoln
D \\ \esoln}  
