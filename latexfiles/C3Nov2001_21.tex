\JGWitem{Nov 2001, \#21}
For a double decrement model:
\begin{enumerate}
\item  In the single decrement table associated with cause (1), $q_{40}^{'(1)}=1/10$ and these decrements are uniformly distributed over the year.
\item  In the single decrement table associated with cause (2), 
$q_{40}^{'(1)}=1/8$ and all decrements occur at time 0.7.
\end{enumerate}
Calculate $q_{40}^{(2)}$
\showMC{\begin{description}
\item[(A)] 0.114
\item[(B)] 0.115
\item[(C)] 0.116
\item[(D)] 0.117
\item[(E)] 0.118
\end{description}}
\showsol{\bsoln
\showMC{\underline{ANS: (C)}}

If decrement occurs due to cause (2), it can only occur at time $t=0.7$. So the $q_{40}^{(2)}$ is the probability that
(40) fails at time $t=0.7$. This is turn is the probability that (40) survives for all $t\in[0,0.7)$  and then fails at that time due to the second cause.

\medskip
So we have to compute the probability that (40) survives for all $t\in[0,0.7)$. 

\medskip
For times $t\in[0.0, 0.7)$, there is only decrement (1) is active.
Since decrement (1) is uniformly distributed over the year and $q_{40}^{'(1)}=1/10$,
we have $\prepostsubsup{t}{q}{40}{'(1)}=t/10$ and $\prepostsubsup{t}{p}{40}{'(1)}=1-t/10$
for all $t\in[0,1]$. Hence the probability of survival up to time $t=0.7$ is $1-0.7/10=0.93$.

\medskip
So \underline{$q_{40}^{(2)}=0.93/8 = 0.11625$}

\esoln}
