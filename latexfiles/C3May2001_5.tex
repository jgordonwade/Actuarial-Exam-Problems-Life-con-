\JGWitem{May 2001, \#5}
For an annual 20-payment whole life insurance of 1000 on (x), you are given:
\begin{enumerate}
 \item $i = 0.06$
 \item $q_{x+19} = 0.01254$
 \item The level (for 20 years) annual benefit premium is 13.72.
 \item The net premium policy value at the end of year 19 is 342.03.
\end{enumerate}
Calculate the level annual benefit premium for an annual whole life insurance of 1000 on (x+20).
\showMC{\begin{description}
\item[(A)] 27
\item[(B)] 29
\item[(C)] 31
\item[(D)] 33
\item[(E)] 35
\end{description}}

\showsol{\bsoln
We're being asked to compute (a thousand times) $P_{x+20}$, which we could do if we knew either of $\ddot{a}_{x+20}$ or $A_{x+20}$ 
(because $A_{x+20} = 1-d\ddot{a}_{x+20}$ and we know $i$). The trick here is recognize that, after 20 years, 
there will be no more premiums, so the benefit reserve $1000(\prepostsub{20}{V}{})$ at that time 
must equal the APV $A_{x+20}$ of the benefit itself at that time, and we're given enough information to 
compute $\prepostsub{20}{V}{}$.

\medskip
Here are the details.
\bears
 1000(\prepostsub{19}{V}{}) + P &=&  1000vp_{x+19}(\prepostsub{20}{V}{}) + 1000vq_{x+19} \\
 342.03 + 13.72 &=& \frac{1000(1-0.01254)}{1.06}(\prepostsub{20}{V}{}) + \frac{1000\cdot 0.01254}{1.06} \\
 1000(\prepostsub{20}{V}{}) &=&  369.184 \\
 A_{x+20} &=& \prepostsub{20}{V}{} =  0.369184 \\
 P_{x+20} &=& \frac{dA_{x+20}}{1-A_{x+20}} = 0.03313 \\
 1000 P_{x+20} &=& 33.13
\eears
 

\esoln}

