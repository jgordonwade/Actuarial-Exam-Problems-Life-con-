\JGWitem{May 2007 \#21}   % \S2 Single Life Survival
The future lifetime random variable for a certain model of building follows a DeMoivre$(\omega)$ (what we in Math 4260 have called the ``$\mbox{gen-DeMoivre}(\omega, 1)$'' distribution).

\bigskip
A new model of building is proposed, with a future lifetime random variable following a $\mbox{gen-DeMoivre}(\omega,a)$ distribution, with the same $\omega.$

\bigskip

You are given the following additional information:
\begin{enumerate}
\item The survival function for one of the new model of buildings is  ${\displaystyle S_0(x)=\left(\frac{\omega-x}{\omega}\right)^a}$ for $x<\omega$. 
\item The new model predicts a $1/3$ higher complete life expectency (over the previous old model with the same $\omega$) for buildings aged 30.
\item The complete life expectency for buildings aged 60 under the new model is 20 years. 
\end{enumerate}
Calculate the complete life expectency of the old model, for buildings aged 70. 

\showsol{\bsoln
If $S_0(x)=[(\omega-x)/\omega]^a$ for $x<\omega$ then
\bears
   \ecirc_x &=& 
\int_{t=0}^{\omega-x} \prepostsub{t}{p}{x}\,dt = \int_{t=0}^{\omega-x} \frac{S_0(x+t)}{S_0(x)}\,dt = \frac{1}{(\omega-x)^a} \int_{t=0}^{\omega-x} (\omega-x-t)^a\,dt \\
            &=& \frac{\omega-x}{a+1}
\eears
With $\ecirc^*_x$ denoting the complete future expection of life of the new model of buuldings, and from the problem statement, we have:
\bears
    \frac{4}{3} &=& \frac{\left[\frac{\omega-30}{a+1}\right]}{\left[\frac{\omega-30}{2}\right]} \quad\Longrightarrow\quad \underline{a+1 = 3/2} \\
    \ecirc^*_{60} &=& 20 = \frac{\omega-60}{a+1} = \frac{2}{3}\cdot(\omega-60) \quad\Longrightarrow\quad \underline{\omega = 90} \\
    \ecirc_{70} &=& (90-70)/2 = 10.
\eears
\underline{ANS: 10 years}

\esoln}
