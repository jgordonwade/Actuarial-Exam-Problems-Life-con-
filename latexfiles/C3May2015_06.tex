\JGWitem{C3 May 2015 \#06} %LTAM Sample Q's (Nov 2019) #8.17
A life insurance company uses the following 3-state Markov model to calculate premiums for a 3-year sickness policy issued to Healthy lives.
\begin{center} Healthy(H) \hspace{1in} Sick(S) \hspace{1in}Dead (D) \end{center}
The company will pay a benefit of 20,000 at the end of each year if the policyholder is Sick at that time.
The insurance company uses the following transition probabilities, applicable in each of the three years:
three years:
\begin{center}\begin{tabular}{c|ccc} & \rule{10pt}{0pt} H \rule{10pt}{0pt}& \rule{10pt}{0pt}S \rule{10pt}{0pt}& \rule{10pt}{0pt}D \rule{10pt}{0pt} \\ \hline
\rule{0pt}{14pt} H & 0.950 & 0.025 & 0.025  \\ 
S & 0.300 & 0.600 & 0.100 \\
D & 0.000 & 0.000 & 1.000
\end{tabular}\end{center}
Calculate the expected present value at issue of sickness benefit payments using an interest rate of 6\%.
\showsol{\bsoln
\bears
  Q &=& \left[\begin{array}{rrr}0.95 & 0.025 & 0.025 \\0.3 & 0.6 & 0.1 \\0 & 0 & 1\end{array}\right], \\
  Q^2 &=& \left[\begin{array}{rrr}0.91 & 0.03875 & 0.05125 \\0.465 & 0.3675 & 0.1675 \\0 & 0 & 1\end{array}\right], \\
  Q^3 &=& \left[\begin{array}{rrr}0.87612 & 0.046 & 0.077875 \\0.552 & 0.23212 & 0.21588 \\0 & 0 & 1\end{array}\right]
\eears
\begin{center}\begin{tabular}{cc} $k$ & Probability that (H) is sick at the end  of $k^{th}$ year \\ \hline  1 &  0.025 \\ 2 & 0.03875 \\ 3 &  0.046 \end{tabular}\end{center}
You can get these probabilities more directy with the path-path approach.

Anyhow:
\[ \text{EPV} = 20,000\cdot \left(\dfrac{0.025}{1.06} + \dfrac{0.03875}{1.06^2} + \dfrac{0.046}{1.06^3}\right) = \boxed{1933.90} \]

\esoln}

%Q=[ 0.950 , 0.025 , 0.025  ; 0.300 , 0.600 , 0.100 ; 0.000 , 0.000 , 1.000 ]


