\JGWitem{May 2000, \#8}
 For a two-year term insurance on a randomly chosen member of a population:
\begin{enumerate}
\item \ $1/3$ of the population are smokers and $2/3$ are nonsmokers.
\item The future lifetimes follow a Weibull distribution with:
\bears
  \tau=2 \quad&&\quad \theta=1.5 \quad \mbox{ for smokers,} \\
  \tau=2 \quad&&\quad \theta=2.0 \quad \mbox{ for nonsmokers.}
\eears
\item The death benefit is 100,000 payable at the end of the year of death.
\item $i = 0.05.$
\end{enumerate}

\bigskip
Calculate the actuarial present value of this insurance.

\bigskip
\showMC{\begin{description}
\item (A) 64,100
\item (B) 64,300
\item (C) 64,600
\item (D) 64,900
\item (E) 65,100
\end{description}}

\showsol{\bsoln
 For each of the two classes of (smokers or nonsmokers) we can use
``first principles'' to compute the
APV of a death benefit payable {\em immediately}, and then the fact that
\[
   \delta \bar{A_x} = iA_x.
\]
Then we can condition on class. 

Here are the details. If $f(t)$ denotes the \pdf of the
future lifetime random variable, we have 
\[
  \bar{A_x} = \int_{0}^{\infty} e^{-\delta t}f(t)\,dt.
\]
From the tables provided with the exam, we have the \pdf:


\esoln}

