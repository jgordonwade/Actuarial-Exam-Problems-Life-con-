\JGWitem{Course 3, Nov 2003, \#4}
Computer maintenance costs for a department are modeled as follows: 
\begin{enumerate}
\item The distribution of the number of maintenance calls each machine will need in a year is Poisson with mean 3. 
\item The cost for a maintenance call has mean 80 and standard deviation 200. 
\item The number of maintenance calls and the costs of the maintenance calls are all mutually independent. 
\end{enumerate}
The department must buy a maintenance contract to cover repairs if there is at least a 10\% probability that aggregate maintenance costs in a given year will exceed 120\% of the expected costs. 
\medskip
Using the normal approximation for the distribution of the aggregate maintenance costs. calculate the minimum number of coniputers needed to avoid purchasing a maintenance contract. 

\showMC{\begin{description}
\item[(A)] 80 
\item[(B)] 90 
\item[(C)] 100 
\item[(D)] 110 
\item[(E)] 120 
\end{description}}

\leaveit{\bsoln

Aggregate repair cost is a compound random variable, call it $S$. 
If there are, say, $n$ computers, then $N\sim\mbox{Poisson}(3n)$. So, 
\bears
  \Expect{S} &=& \Expect{N}\Expect{X} = 3n\times 80 = 240n \\
  \Var{S} &=& \Expect{N}\Var{X} + \Expect{X}^2\Var{N} = 3n(200^2 + 80^2) = (373.1\sqrt{n})^2. 
\eears
\bears
  && \Prob{S>1.2\Expect{S}} = 0.1   \quad\Longrightarrow \quad
     \Prob{S\leq 288n} = 0.9
  \\  \Longrightarrow &&
     \Prob{\frac{S-240n}{373.1\sqrt{n}}\leq \frac{288n-240n}{373.1\sqrt(n)}} = 0.9
  \\  \Longrightarrow &&
     \Phi\left(\frac{48\sqrt{n}}{373.1}\right) = 0.9
  \quad \Longrightarrow \quad
     \frac{48\sqrt{n}}{373.1} = 1.28155  \quad\Longrightarrow \quad
      n = 99.229
\eears
\underline{Ans: C}

\esoln}



