\JGWitem{ LTAM May 2020, \#14}
For a fully discrete whole life insurance of 1000 issued to (60), you are given the
following information:
\begin{itemize}
  \item[(i)] Renewal expenses are 5 per year.
  \item[(ii)] Commissions are 10\% of the gross premium.
  \item[(iii)] Mortality follows the Standard Ultimate Life Table.
  \item[(iv)] The valuation interest rate is 3\% per year for the first 10 years, and 5\% per year thereafter.
  \item[(v)] The gross annual premium is 36.
\end{itemize}

Calculate the gross premium reserve at the end of the 9th year.
\showsol{\bsoln
\[  _9\!V = \text{\underline{EPV future benefits}} + \text{\underline{EPV future expenses}} - \text{\underline{EPV future premiums.}} \]

\begin{itemize}
\item \underline{EPV future benefits.} Denote this by $A$. Be careful to account for the changing interest rate. 
From the SULT we have $\Ax{70}$ computed at 5\% is 0.4281760, and so
\bears A &=& \dfrac{1000\qx{69}}{1.03} + \dfrac{1000\cdot \px{69}}{1.03}0.4281760 \\
       &=&\dfrac{1000(1-0.9907061)}{1.03} + \dfrac{1000\cdot 0.9907061}{1.03}\cdot 0.4281760 = \underline{420.8645}
\eears
\item \underline{EPV future expenses.} Denote this by $E$, denote the gross annual premium by $P$ (which is 36),
  and denote the EPV of the life annuity on (69) by $a$. Again we have to be careful to account for the changing interest rate. 
  From the SULT we have $\ax**{70}$ computed at 5\% is 12.008303, and so
  \bears
     E &=& (0.1\cdot P + 5)\cdot a = 8.6\cdot a \\
     && a = 1+\dfrac{\px{69}}{1.03}\cdot \ax**{70} = 1+\dfrac{0.9907061}{1.03}\cdot 12.008303 = = \underline{12.550193} \\
     E &=& 8.6\cdot 12.550193 = \underline{107.93166}
  \eears
\item \underline{EPV future Premiums:} Denote this by $Y$, then $Y=P\cdot a = 36\cdot  12.550193= \underline{451.806948}$
\end{itemize}
Then
\[  \underline{_9\!V = 420.8645 + 107.93166 - 451.806948   = \boxed{76.989}} \]

\hrulefill

SHORTER SOLUTION via the the recursive formula 
\[(_9\!V + P-\text{Expenses})\cdot(1+i) = \qx{69}\text{(EPV of Benefit)} + \px{69}\cdot(_{10}\!V),\]
 which is shorter because $_{10}\!V$ is relatively easy to compute:
\bears
 _{10}\!V &=& 1000\cdot \Ax{70} + (0.1\cdot 36 + 5)\ax**{70} - 36\ax**{70} = 1000\cdot \Ax{70}  - 27.4\ax**{70}  
          \\ &=& 1000\cdot 0.4281760 - 27.4\cdot 12.008303 = \underline{99.14850} \\
   _9\!V &=& \dfrac{1}{1.03}\Bigl[1000\qx{69} 1000 + \px{69}\cdot \left(_{10}\!V\right)\Bigr] - P + (0.1P+5) \\
   _9\!V &=& \dfrac{1}{1.03}\left( 1000(1-0.9907061)  +  0.9907061\cdot 99.14850\right) - 27.4 = \boxed{76.989}
\eears
\esoln}

