\JGWitem{Course M, Nov 2006, \#40}
A compound Poisson distribution has $\lambda = 5$ and claim amount distribution as follows: 
\begin{center}\begin{tabular}{cc}
  $x$ &  $p(x)$ \\ \hline 
  100 & 0.80  \\
  500 & 0.16  \\
  1000 & 0.04 
\end{tabular}\end{center}
Calculate the probability that aggregate claims will be exactly 600. 
\showMC{\begin{description}
\item[(A)] 0.022 
\item[(B)] 0.038 
\item[(C)] 0.049 
\item[(D)] 0.060 
\item[(E)] 0.070 
\end{description}}

\leaveit{\bsoln
\bears
  \Prob{S=600} &=& \sum_{n=0}^\infty \Prob{S=600|N=n}\Prob{N=n} \\
  &=& \Prob{S=600|N=2}\Prob{N=2} \\ 
  &&\quad + \quad\Prob{S=600|N=6}\Prob{N=6} \\
  &=& 2\times 0.8\times 0.16\times\left(\frac{5^2e^{-5}}{2!}\right) + 0.80^6\times \left(\frac{5^6e^{-5}}{6!}\right) \\
  &=& 0.05989
\eears
(The factor of 2 in the first term reflects the fact that there could be a 100 then a 500, or a 500 then a 100.)

\underline{Ans: D}

\esoln} 


