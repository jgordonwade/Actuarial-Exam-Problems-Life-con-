\JGWitem{C3 Nov 2017 \#03}  %LTAM Sample (Nov 2019) #8.24
You are given the following Markov chain model:
\begin{itemize}
\item[(i)] Annual transition probabilities between the states Healthy, Sick and Dead, of an organism are as follows:
\begin{center}\begin{tabular}{c|ccc} & Healthy & Sick & Dead \\ \hline
                       Healthy & 0.64 & 0.16 & 0.20 \\ Sick & 0.36 & 0.24 & 0.40 \\ Dead & 0 & 0 & 1  \end{tabular}\end{center}
\item[(ii)] Transitions occur at the end of the year.
\end{itemize}
A population of 1000 organisms starts in the Healthy state. Their future states are independent.

\bigskip
Using the normal approximation without the continuity correction, calculate the probability that there will be at least 625 organisms alive (Healthy or Sick) at the beginning of the third year.
\showsol{\bsoln
  The probability that there will be at least 625 organisms alive at the beginning of the third year is the same as
  the probability that at most  375 organisms have died by the beginning of the third year. If we let $N$ denote the number
  of organisms that have died by the beginning of the third year, then we see $\Prob{N\leq 375}.$ Since the organisms' lives are independent,
  $N\sim\text{Binomial}(1000,p)$  for some $p$.

  To compute $p$, we consider a single Healthy organism, for which the probabilityof being Dead at the beginning of the third year is 
  \bears 
     \text{H}\rightarrow\text{H}\rightarrow\text{D} && 0.64\cdot 0.2 \\
     \text{H}\rightarrow\text{S}\rightarrow\text{D} && 0.16\cdot 0.4 \\
     \text{H}\rightarrow\text{D}                              && 0.2        \\ && \rule{50pt}{0.1pt} \\
             && p=0.392
   \eears
  For a Binomial(1000,0.392) random variable, 
  \bears
         \mu &=& np = 392, \\
         \sigma &=& \sqrt{np(1-p)} = 15.4381
   \eears

   \bigskip
   Since $n=1000$ is reasonably large, the normal approximation makes sense: If
  \[  Z=\dfrac{N-\mu}{\sigma} \]
  then $Z$ is approximately $\text{Normal}(0,1)$-distributed, and 
  \bears
     \Prob{N\leq 375} &\approx& \Prob{Z\leq z} = \Phi(z), \qquad\text{ where }\qquad z =  \dfrac{372-\mu}{\sigma}  = -1.1012 \\
                  &=& \Phi(-1.1012)  = \boxed{ 0.1354}
   \eears
\esoln}

%(0.64+0.16+1)*0.2
%Q=[ 0.64  0.16  0.20 ;  0.36  0.24  0.40 ;   0  0  1  
