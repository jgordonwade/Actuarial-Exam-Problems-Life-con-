\JGWitem{Nov 2000, \#24 (book $6^{th}$ ed., \#13.1}
For students entering a three-year law school, you are given:
\begin{enumerate}
\item The following double decrement table:
\begin{center}\begin{tabular}{|c|c|c|c|}\hline
                 & \multicolumn{3}{|c|}{For a student at the beginning of that academic year, Probability of: } \\ \hline
  Academic Year  & Academic  & Withdrawal for    & Survival Through    \\ 
                 & Failure   & All Other Reasons & Academic Year       \\ \hline \hline
       1         & 0.40             & 0.20                             & ---  \\ \hline                
       2         &  ---             & 0.30                             & ---  \\ \hline   
       3         &  ---             & ---                              & 0.60 \\ \hline   
\end{tabular}\end{center}
\item Ten times as many students survive year 2 as fail during year 3.
\item The number of students who fail during year 2 is 40\% of the number of students who survive year 2.
\end{enumerate}
Calculate the probability that a student entering the school will withdraw for reasons
other than academic failure before graduation.
\showMC{\begin{description}
\item[(A)] Less than 0.35
\item[(B)] At least 0.35, but less than 0.40
\item[(C)] At least 0.40, but less than 0.45
\item[(D)] At least 0.45, but less than 0.50
\item[(E)] At least 0.50
\end{description}}
\showsol{\bsoln
\showMC{\underline{ANS: (B)}}
Let's call ``academic failure'' decrement (1), and ``withdrawal for other reasons'', decrement (2),
let $l_0$ be the number of students entering law school, and let $l_x$ the number still in school after x years. 


Consider the given coniditon (b). The number of students surviving year 2 is the same as the number that make it to year 3, which is $l_3$. 
And the number of student who fail in year 3 is the number that make it to year 3 times the percentage failing that year, 
which is $l_3q_2^{(1)}$.  So given (b) is saying:
\bears
   l_3 &=& 10(l_3\cdot q_2^{(1)}) \quad\quad \Longrightarrow \underline{(i): \; \; q_2^{(1)} = 0.10} \\
\eears
Further:
\bears
   p_0^{(\tau)} &=& 1 - q_0^{(\tau)} = 1 - (q_0^{(1)} + q_0^{(2)}) = \underline{0.40 = p_0^{(\tau)} \;\; (ii)} \\ \\
   l_1 &=& (l_0)p_0^{(\tau} = 0.40\,l_0. \\
   \mbox{From (c): } l_1q_1^{(1)} &=& 0.40\cdot(l_1p_1^{(\tau)}) \\
   q_1^{(1)} &=& 0.40\cdot (1-q_1^{(1)} - q_1^{(2)}) = 0.40\cdot (1-q_1^{(1)} - 0.30) \\
   && \quad\quad\Longrightarrow \underline{(iii): \;\; q_1^{(1)} = 0.20} \\ \\
   1 &=& q_1^{(1)} + q_1^{(2)} + p_1^{(\tau)} 
    \quad\quad\Longrightarrow \underline{(iv): \;\; p_1^{(\tau)} = 0.50} \\ \\
   1 &=& q_2^{(1)} + q_2^{(2)} + p_2^{(\tau)} 
     = 0.1 + q_2^{(2)} + 0.6 
    \quad\quad\Longrightarrow \underline{(v): \;\; q_2^{(2)} = 0.30} \\ 
\eears
%\showprob{\newpage}
So we've got our table filled in:
\begin{center}\begin{tabular}{|c|c|c|c|}\hline
  Academic Year  & Academic  & Withdrawal for    & Survival Through    \\ 
                 & Failure   & All Other Reasons & Academic Year       \\ \hline \hline
       1         & 0.40             & 0.20                             & (ii) 0.40  \\ \hline                
       2         & (iii) 0.20       & 0.30                             & (iv) 0.50  \\ \hline   
       3         & (i) 0.10         & (v) 0.30                         & 0.60 \\ \hline   
\end{tabular}\end{center}
Now
\bears
   \prepostsubsup{2}{q}{0}{(\tau)} &=& q_0^{(2)} + (p_0^{(\tau)})(q_1^{(1)}) + (\prepostsubsup{2}{p}{0}{(\tau)})q_2^{(2)} \\
        &=& 0.20 + (0.40)\cdot(0.30) + (0.4)\cdot(0.5)\cdot(0.30) = 0.3800.
\eears

\esoln}
