\JGWitem{Nov 2003, \#11}
 For a fully discrete whole life insurance of 1000 on (60), the annual 
benefit premium was calculated using the following:
\begin{enumerate}
\item $i=0.06$
\item $q_{60}=0.01376$
\item $1000 A_{60}=369.33$
\item $1000 A_{61}=383.00$.
\end{enumerate}
A particular insured is expected to experience a first-year mortality 
rate ten times the rate used to calculate the annual benefit premium. 
The expected mortality rates for all other years are the ones originally used.

\medskip
Calculate the expected loss at issue for this insured, based on the original benefit premium.
\begin{description}
\item[(A)] 72
\item[(B)] 86
\item[(C)] 100
\item[(D)] 114
\item[(E)] 128
\end{description}
\bsoln The original premium is $1000P_{60}$, 
\[
  P_{60}={A_{60}\over \ddot{a}_{60}}={dA_{60}\over 1-A_{60}} 
  = {(0.06/1.06)\cdot 0.36933\over 1-0.36933} = 0.033148.
\]
The whole life benefit APV $A_{60}$ can be written as a one-year term benefit APV
plus a one-year deferred whole life benefit APV:
\bears
  A_{60} &=& A_{60:\halfbox{1}}\!\!\!\!\!\!\!\!\raisebox{0.80ex}{$\scriptscriptstyle 1$}\;\;\;
             + E_{60}A_{61} \\
         &=& vq_{60} + v(1-q_{60})A_{61}.
\eears
The first-year mortality rate $q_{60}$ is magnified by 10. 
\bears
  A_{60}^{{\scriptscriptstyle NEW}} &=& 10 vq_{60} + v(1-10 q_{60})A_{61} \\
  &=& \bigl(0.1376 + (1-0.1376)\cdot 0.383\bigr)/1.06 = 0.44141 \\
  \Expect{L}&=& \Bigl(1000+{P\over d}\Bigr)A_{60}^{{\scriptscriptstyle NEW}}
   - {P\over d} \\
  &=& (1000+585.62)\cdot 0.44141 - 585.62\\
  &=& 114.29
\eears
ANS: \underline{D}

\esoln
