\JGWitem{May 2000, \#5}
 An insurance company has agreed to make payments to a worker age 
x who was injured at work.
\begin{enumerate}
\item The payments are 150,000 per year, paid annually, starting immediately 
and continuing for the remainder of the worker's life.
\item After the first 500,000 is paid by the insurance company, 
the remainder will be paid by a reinsurance company.
\item 
 \[
   \prepostsub{t}{p}{x} = \left\{\begin{array}{cc}
     (0.7)^t & \mbox{ for }0\leq t \leq 5.5, \\
      0      & \mbox{ otherwise.}
   \end{array}\right.
\]
\item $i = 0.05$
\end{enumerate}

\bigskip
Calculate the actuarial present value of the payments to be made by the 
reinsurer.

\showMC{\begin{description}
\item (A) Less than 50,000
\item (B) At least 50,000, but less than 100,000
\item (C) At least 100,000, but less than 150,000
\item (D) At least 150,000, but less than 200,000
\item (E) At least 200,000
\end{description}}

\showsol{\bsoln
If the worker survives for three years, the reinsurance will pay 100,000 at time $t=3$ and everything after that.
So the actuarial present value of the reinsurer's portion of the claim is
\[
   100,000\times\left(\frac{0.7}{1.05}\right)^3 
 + 150,000\times\left[ \left(\frac{0.7}{1.05}\right)^4 + \left(\frac{0.7}{1.05}\right)^5 \right] = 79,012.
\]
\esoln}
