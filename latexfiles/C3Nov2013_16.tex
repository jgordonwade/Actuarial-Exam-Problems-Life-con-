\JGWitem{C3, Nov 2013, \#16}
For fully discrete whole life insurances of 1 issued on lives age 50, the annual net premium, P, was calculated using the following:
\begin{enumerate}
\item $q_{50}=0.0048$
\item $i=0.04$
\item $A_{51}=0.39788$.
\end{enumerate}
A particular life has a first year mortality rate 10 times the rate used to calculate P. The mortality rates for all other years are the same as the ones used to calculate P.

\medskip
Calculate the expected present value of the loss at issue random variable for this life, based on the premium P.

\showsol{\bsoln
  The loss random variable (``loss at issue'') is given by
  \[
       L = Z_{50} - \dfrac{P}{d}(1-Z_{50})
  \]
  and since $P$ was computed via the Equivalance Principle (because ``net premium''), we can set $\Expect{L}=0$ and solve for $P$, yielding
 \[ P=(dA_{50}/(1-A_{50}), \]
 so $L$ becomes (after simplifying)
\[    L = \dfrac{Z_{50}-A_{50}}{1-A_{50}}, \]
which equals zero because $\Expect{Z_{50}}=A_{50}$. 

\bigskip
The fact that ``a particular life has a first year mortality rate'' changes means
that $Z_{50}$ changes to, say, $Z_{50}^*\neq Z_{50}$, and $\Expect{Z_{50}^*}\neq A_{50}$.
Instead, since $q_{50}$ has changed to $10q_{50} = 0.048$, we have
\bears
    \Expect{Z_{50}^*} &=& vq_{50}^* + vp_{50}^*A_{51}
    = \dfrac{0.048}{1.04} + \dfrac{1-0.048}{1.04}\cdot 0.39788 = 0.4103\,6708.
\eears
Also, $A_{50} = v\qx{x}+v\px{x}A_{51}=0.0048/1.04 + (1-0.0048)\cdot 0.39788/1.04 = 0.3853\,5594$. So
\[
    \Expect{L} = \Expect{\dfrac{Z_{50}-A_{50}}{1-A_{50}}} = \dfrac{0.4103\,6708 - 0.3853\,5594}{1-0.3853\,5594} = \boxed{0.04069}
\]
\esoln}
