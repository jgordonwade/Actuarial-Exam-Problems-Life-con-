\JGWitem{ LTAM May 2020, \#16 }
An insurer issues fully discrete whole life policies of 100,000 to a group of lives age (60).
The annual premium for each policy is 3765.

\smallskip
At the end of year 5, there were 100 policies still in force. In the following year:
\begin{itemize}
  \item[(i)] One policyholder died;
  \item[(ii)] Expenses of 11\% of each premium paid were incurred;
  \item[(ii)] Actual interest earned was 5\%.
\end{itemize}
You are also given that reserves per policy at the end of the 5$^{th}$ and 6$^{th}$ years are
$_5\!V = 11,190$ and $_6\!V = 13,529,$ respectively.

\medskip
Calculate the profit on this group of policies in the 6th year.

\showsol{\bsoln
  Reserves in consecutive years are related via the recursion formula
  \[  \underbrace{\Bigl(\text{(Old reserve)} + \text{(Prem. collected)} - \text{(Exp. Incurred)}\Bigr)(1+i)}_{\text{funds on hand at end of $6^{th}$ year}}
      = \left\{\begin{array}{cl} \text{Death Benefit} & \text{ if (x) dies} \\ \text{New reserve} & \text{ if (x) lives}. \end{array}\right.\]

   In this case, for policies in which the insured dies in the $6^{th}$ years, this yields
  \[  \Bigl( 11,190+ 3765(1- 0.11) \Bigr)1.05 - 100,000 = -84,732.11, \]
      and for policies in which the insured survives the $6^{th}$ years, this yields
  \[  \Bigl( 11,190+ 3765(1- 0.11) \Bigr)1.05 - 13,529 = 1738.89, \]
  so if one person dies and 99 survives, then the profit that year is 
  \[  99 \cdot 1738.89 - 84,732.11 = \boxed{87,418.00} \]
     

\esoln}



   

