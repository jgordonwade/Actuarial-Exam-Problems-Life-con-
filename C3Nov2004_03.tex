\JGWitem{Nov 2004, \#3}
For a fully continuous whole life insurance of 1 on (x), you are given: 
\begin{enumerate}
\item The forces of mortality and interest are constant, 
\item $\pretwo{\overline{A}}_x=0.20$, 
\item The net level annual premium for a unit whole life insurance paying immediately upon death is $0.03.$ %$\overline{P}(\overline{A}_x) = 0.03$, and, 
\item $\prepostsub{0}{L}{}$ is the loss-at-issue random variable based on the 
 benefit premium
\end{enumerate}
Calculate $\Var{\prepostsub{0}{L}{}}$. 
\showMC{\begin{description}
\item[(A)] 0.20
\item[(B)] 0.21
\item[(C)] 0.22
\item[(D)] 0.23
\item[(E)] 0.24
\end{description}}
\showsol{\bsoln
Since $\delta$ and $\mu$ are constant we have 
$\overline{A}_x=\mu/(\mu+\delta)$ and $\overline{a}_x = 1/(\mu+\delta)$ 
so that the equivalence principle becomes
\[ 0 = \dfrac{\mu}{\mu+\delta} -\dfrac{P}{\mu+\delta}, \]
so \underline{$\mu=0.03$} according to the given data. 

\medskip
Then $\pretwo{\overline{A}}_x=0.20$ and 
$\pretwo{\overline{A}}_x=0.03/(0.03+2\delta)$ mean that 
$0.03/(0.03+2\delta)=0.20$ so that $\underline{\delta=0.06}$. 

\medskip 
We'll also need: $\overline{A}_x=\mu/(\mu+\delta) = 0.03/0.09 = 1/3$.

\medskip
Now, 
\bears
  \prepostsub{0}{L}{} &=& \overline{Z}_x + P\overline{Y}_x 
     = \left(1+\frac{P}{\delta}\right)\overline{Z}_x  - \frac{P}{\delta} \\
  \Var{\prepostsub{0}{L}{}} &=& \left(1+\frac{P}{\delta}\right)^2\Var{\overline{Z}_x} 
   = \left(1+\frac{P}{\delta}\right)^2\left[{\pretwo{\overline{A}}_x - \overline{A}_x^2}\right] \\
   &=& \left(1+\frac{0.03}{0.06}\right)^2\times(0.20 - (1/3)^2) = 0.20.
\eears
\esoln}


%Holiday Inn 
%3700 Central Ave.
%139.00/night
%conf. # 612 605 04
