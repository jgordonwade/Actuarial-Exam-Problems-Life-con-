\JGWitem{Nov 2000, \#23} 
 Workers' compensation claims are reported according to a Poisson process with mean
100 per month. The number of claims reported and the claim amounts are independently
distributed. Two percent of the claims exceed 30,000.
\medskip
Calculate the number of complete months of data that must be gathered to have at least a
90\% chance of observing at least 3 claims each exceeding 30,000.
\showMC{\begin{quote}\begin{description}
\item[A] 1
\item[B] 2
\item[C] 3
\item[D] 4
\item[E] 5
\end{description}\end{quote}}
\showsol{\bsoln
 Whether a claim exceeds 30,000 or not can be represented by a Bernoulli(p) distribution
with $p=0.02$. The number of claims exceeding 30,000 in $n$ months is a compound
Poisson with frequency $\lambda 100n$ and Bernoulli(p) severity. We have
\bears
  M_X(t) &=& 1-p+pe^t, \\
  M_S(t) &=& e^{\lambda(M_X(t)-1)} = e^{\lambda p(e^t-1)} \\
  &=& \mbox{The \mgf for a Poisson$(\lambda p)$ distribution} \\
  \Prob{S=k}&=& {(\lambda p)^k e^{-\lambda p}\over k!} \\
  \Prob{S=k}&=& {(2n)^k e^{-2n}\over k!} \\
  \Prob{S>2} &=& 1-(\Prob{S=0}+\Prob{S=1}+\Prob{S=2}) \\
  \ldots
\eears
\esoln}
