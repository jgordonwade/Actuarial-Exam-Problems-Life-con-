\JGWitem{ LTAM May 2020, \#3 }
For a special fully discrete 3-year term insurance with level premiums on (x), you are given:
\begin{itemize}
  \item[(i)] The death benefits, $b_{k+1}$, and mortality rates are:
     \begin{center}\begin{tabular}{ccc}
             $k$ & $b_{k+1}$ & $q_{x+k}$ \\
               0 & 100,000 & 0.03 \\
               1 & 200,000 & 0.05 \\
               2 & 300,000 & 0.07   \end{tabular}\end{center}
   \item[(ii)] $i=0.06$
\end{itemize}
Calculate the net premium reserve at the end of the first policy year.
   

\showsol{\bsoln
  We have to determine the premium using the equivalence principle, and since
  the benefit is varying we have to use the basic definition of expected value $Z$ of the PV of the benefit,
  \dsy{     \Expect{Z} = \sum_{k=0}^2 b_{k+1}v^{k+1}\qx[k|]{x}. }
  So we need the values of $\qx[k|]{x}$, then compute $\Expect{Z}$ and also the expected present value $\Expect{Y}$ of the premiums,
  and then apply the equivalence principle.
  \bears
     \qx[0|]{x} &=& \qx{x} = 0.03 \\
     \qx[1|]{x} &=& \px{x}\qx{x+1} = 0.97\cdot 0.05  =0.0485 \\
     \qx[2|]{x} &=& \px[2]{x}\qx{x+2} = 0.97\cdot 0.95\cdot 0.07 = 0.064505 \\
     \Expect{Z} &=& 100,000\cdot\left( \dfrac{0.03}{1.06} + 2\dfrac{0.0485}{1.06^2} + 3\dfrac{0.064505}{1.06^3} \right) = 27,711.05 \\
     \Expect{Y} &=& \sum_{k=0}^2 \Ex[k]{x} = 1 + \dfrac{(1-0.03)}{1.06} + \dfrac{(1-0.03)(1-0.05)}{1.06^2} = 2.73522606, \\
     0 &=& \Expect{Z} - P\Expect{Y} \quad\quad\Longrightarrow\quad \underline{P = 10,131.20}
  \eears
  Now we can compute $_1\!V$ from the defintion,  $ _1\!V = \Expect{_1!L}$, but it's faster to use the recursion formula:
  \bears
      (_0\!V + P)\cdot 1.06 &=& \qx{x}\cdot 100,000 + \px{x}\cdot \left(_1\!V\right) \\
      (0 + 10,131.20)\cdot 1.06 &=& 0.03\cdot 100,000 + (1-0.03)\left(_1\!V\right) \quad\quad\Longrightarrow\quad \boxed{_1\!V=7978.42}
  \eears
\esoln}  


