\JGWitem{May 2005 \#25}
 Beginning with the first full moon in October deer are hit by cars at a 
Poisson rate of 20 per day. The time between when a deer is hit and when 
it is discovered by highway maintenance has an exponential distribution 
with a mean of 7 days. The number hit and the times until
they are discovered are independent.

Calculate the expected number of deer that will be discovered in the first 
10 days following the first full moon in October.
\showMC{\begin{description}
\item[(A)] 78
\item[(B)] 82
\item[(C)] 86
\item[(D)] 90
\item[(E)] 94
\end{description}}
\leaveit{\bsoln
The Poisson process of deer-hits can be decomposed into two processes: Deer hit but not found, and deer hit and found. 

If a single deer is hit before time $t=10$, then what is the probability that it is found by time $t=10$? With $T$ the wait-time
until the dead deer is found, we have:
\bears
  \Prob{\mbox{fnd by $t=10$}} &=& \int \Prob{\mbox{fnd by $t=10$ $|$ hit at time $s$}}\Prob{\mbox{hit at time $s$}}\\
        &=& \int_0^{10} F_T(10-s)\frac{1}{10}\,ds \\
        &=& \frac{1}{10}\int_0^{10} 1-e^{(s-10)/7}\,ds = \frac{3}{10} + \frac{7}{10}e^{-10/7} \approx 0.4678.
\eears
So, hit-and-found deer happen at a Poisson rate of $0.4678\times 20\times 10 = 93.56$

\underline{Ans: E}

\esoln}
