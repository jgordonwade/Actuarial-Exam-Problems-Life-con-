\JGWitem{Nov 2001, \#20}
Don, age 50, is an actuarial science professor. His career is subject to two decrements:
\begin{enumerate}
\item Decrement 1 is mortality. The associated single decrement table follows De Moivre’s
law with $\omega=100$.
\item  Decrement 2 is leaving academic employment, with $\mu_{50}^{(2)}=0.05$ for $t\geq 0.$
\end{enumerate}
Calculate the probability that Don remains an actuarial science professor for at least five but
less than ten years.
\showMC{\begin{description}
\item[(A)] 0.22
\item[(B)] 0.25
\item[(C)] 0.28
\item[(D)] 0.31
\item[(E)] 0.34
\end{description}}
\showsol{\bsoln
\showMC{\underline{ANS: (A)}}

The quantity we seek is $\prepostsubsup{5|5}{q}{50}{(\tau)}$ which we can write cleanly in terms of
the survival function, so let's get the survival function, $\prepostsubsup{t}{p}{50}{(\tau)}$.  

\medskip
The phrase ``asssociated single decrement table follows deMoivre's law'' indicates that
\dsy{\prepostsubsup{t}{p}{50}{'(1)} = 1-\frac{t}{50}}, and the fact that decrement (2) has a constant hazard rate
indicates that \dsy{\prepostsubsup{t}{p}{50}{'(2)} = e^{-t/20}.} So
\bears
   \prepostsubsup{t}{p}{50}{(\tau)} &=& \left(\prepostsubsup{t}{p}{50}{'(1)}\right)\left(\prepostsubsup{t}{p}{50}{'(2)}\right)
         = \frac{1}{50}(50-t)e^{-t/20}, \\
   \prepostsubsup{5|5}{q}{50}{(\tau)} &=&  \prepostsubsup{5}{p}{50}{(\tau)} - \prepostsubsup{10}{p}{50}{(\tau)} \\
                    &=& \frac{1}{50}\left( 45e^{-5/20} - 40e^{-10/20} \right)  = \underline{0.215696}
 \eears
 \esoln}
